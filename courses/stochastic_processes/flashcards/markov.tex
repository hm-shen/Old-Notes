% layout in Anki:
\documentclass[11pt]{article}
\usepackage[a4paper]{geometry}
\geometry{paperwidth=.5\paperwidth,paperheight=100in,left=2em,right=2em,bottom=1em,top=2em}
\pagestyle{empty}
\setlength{\parindent}{0in}

% hyphenation:
\usepackage[ngerman]{babel}

% encoding:
\usepackage[T1]{fontenc}
\usepackage[utf8]{inputenc}
\usepackage{lmodern}

% packages:
\usepackage{amsmath,amsfonts}

%%%%%%%%%%%%%%%%%%%%%%%%%%%%%%%%%%%%%%%%%%%%%%%%%%%%
% Following part of header NOT to be copied into
%            the note options in Anki.
%          ! Anki will throw an errow !
%%%%%%%%%%%%%%%%%%%%%%%%%%%%%%%%%%%%%%%%%%%%%%%%%%%%%
%
%  pdf layout:
%
\geometry{paperheight=74.25mm}
\usepackage{pgfpages}
\pagestyle{empty}
\pgfpagesuselayout{8 on 1}[a4paper,border shrink=0cm]
\makeatletter
\@tempcnta=1\relax
\loop\ifnum\@tempcnta<9\relax
\pgf@pset{\the\@tempcnta}{bordercode}{\pgfusepath{stroke}}
\advance\@tempcnta by 1\relax
\repeat
\makeatother
%
%  notes, fields, tags:
%
\newcommand{\xfield}[1]{
        #1\par
        \vfill
        {\tiny\texttt{\parbox[t]{\textwidth}{\localtag\\\globaltag\hfill\uuid}}}
        \newpage}
\newenvironment{field}{}{\newpage}
\newif\ifnote
\newenvironment{note}{\notetrue}{\notefalse}
\newcommand{\localtag}{}
\newcommand{\globaltag}{}
\newcommand{\uuid}{}
\newcommand{\tags}[1]{
    \ifnote
        \renewcommand{\localtag}{#1}
    \else
        \renewcommand{\globaltag}{#1}
    \fi
    }
\newcommand{\xplain}[1]{\renewcommand{\uuid}{#1}}
%
%%%%%%%%%%%%%%%%%%%%%%%%%%%%%%%%%%%%%%%%%%%%%%%%%%%%
% The following line again needs to be copied
% into Anki:
\begin{document}
%%%%%%%%%%%%%%%%%%%%%%%%%%%%%%%%%%%%%%%%%%%%%%%%%%%%

\begin{note}
   \begin{field}
       Definition of Renewal Processes
   \end{field}
   \begin{field}
     A renewal process is a counting process where inter-arrival
     times, \(\tau_j, j = 1,2,\ldots\) between consecutive events are
     independent identically distributed with cumulative distribution
     function \(F_\tau(\cdot)\), \(\mu := \mathbb{E}(\tau)\),
     \(\tau > 0\) with probability \(1\).
   \end{field}
\end{note}
%
\begin{note}
  \begin{field}
    Monotone convergence Theorem
  \end{field}
  \begin{field}
    Let \(X_n, n = 1,2,\dots\) be a sequence of random variables such
    that \(\forall n \geq 1, 0 \leq X_n \leq X_{n+1}\) with
    probability \(1\). Then
    \(\lim_{t \to +\infty} \mathbb{E}(X_n) = \mathbb{E}(\lim_{n \to
      +\infty} X_n)\).
  \end{field}
\end{note}
%
\begin{note}
  \begin{field}
    Properties of Renewal Processes \(\{N_t, t \geq 0\}\)
  \end{field}
  \begin{field}
    \begin{enumerate}
    \item For any fixed finite \(t\), \(P(N_t = +\infty) = 0\).
    \item \(\lim_{t \to +\infty} N_t = +\infty\) with probability \(1\).
    \item \(\lim_{t \to +\infty} \mathbb{E}(N_t) = +\infty\).
    \end{enumerate}
  \end{field}
\end{note}
%
\begin{note}
  \begin{field}
    Strong law of renewal processes
  \end{field}
  \begin{field}
    \(\lim_{t \to +\infty} \frac{N_t}{t} = \frac{1}{\mu}\) with
    probability \(1\).
  \end{field}
\end{note}
%
\begin{note}
  \begin{field}
    The distribution of a renewal process, \(\{N_t, t \geq 0\}\) with
    inter-arrival times \(\tau_n, n = 1,2,\ldots\), (the distribution
    of \(\tau_n\) is given: \(F_\tau(\cdot)\)
  \end{field}
  \begin{field}
    \(P(N_t = n) = F_n(t) - F_{n+1}(t)\), where
    \(F_n = F_\tau * F_\tau \cdots * F_\tau\)
  \end{field}
\end{note}
%
\begin{note}
  \begin{field}
    Central limit theorem for renewal processes
  \end{field}
  \begin{field}
    Assume that the inter-arrival times, \(\tau_j\) for a renewal
    process, \(\{N_t, t \geq 0\}\) have finite variance
    \(\sigma^2\). Then
    \(\xi_t = \frac{N_t - \frac{t}{\mu}}{\sigma \sqrt{t / \mu^3}} \to
    \mathcal{N}(0,1)\) in distribution when \(t \to +\infty\).
  \end{field}
\end{note}
%
\begin{note}
  \begin{field}
    The renewal equation
  \end{field}
  \begin{field}
    \(m(t) = F_\tau(t) + \int_0^t m(t-x) F(dx)\)
  \end{field}
\end{note}
%
\begin{note}
  \begin{field}
    The elementary renewal theorem (\(\tau\) is cont.)
  \end{field}
  \begin{field}
    \(\lim_{t \to +\infty} \frac{\mathbb{E}(N_t)}{t} =
    \frac{1}{\mathbb{E}(\tau)}\)
  \end{field}
\end{note}
%
\begin{note}
  \begin{field}
    Blackwell's Theorem
  \end{field}
  \begin{field}
    If \(\{N_t, t = 0,1,\ldots\}\) has an inter-arrival distribution
    that is no-lattice, then \(\forall \sigma > 0\),
    \begin{align*}
      \lim_{t \to +\infty} [ m(t+\sigma) - m(t) ] = \frac{\sigma}{\mu}
      \\
      \lim_{t \to +\infty} P(N_{t + \sigma} - N_t = 1) =
      \frac{\sigma}{\mu} \\
      \lim_{t \to +\infty} P(N_{t+\sigma} - N_t = 0) = 1 -
      \frac{\sigma}{\mu} + o(\sigma) \\
      \lim_{t \to +\infty} P(N_{t+\sigma} - N_t = 2) = o(\sigma)
    \end{align*}
    If \(\{N_t, t = 0,1,\ldots\}\) has an inter-arrival distribution
    that is lattice with span \(d\), then:
    \begin{align*}
      \lim_{t \to +\infty} [ m(nd) - m((n-1)d)] = \frac{nd}{\mu}
    \end{align*}
  \end{field}
\end{note}
%
\begin{note}
  \begin{field}
    Definition of Markov Chain, how to describe a Markov Chain
  \end{field}
  \begin{field}
    A Markov chain is a Markov process \(\{X_t, t \in T\}\) with
    finite or countably infinite state space. To characterize a Markov
    chain statistically, we need the following
    \begin{align*}
      \{\pi_s(x) : \forall x \in S, \pi_0(x) = p(X_0 = 0)\} \\
      \{P_t(x, y) = p(X_{t+1} = y | X_t = x), x, y \in S, t \in T\}
    \end{align*}
    where \(S\) is the state of space of the Markov chain.
  \end{field}
\end{note}

\end{document}

%%% Local Variables:
%%% mode: latex
%%% TeX-master: t
%%% End:
