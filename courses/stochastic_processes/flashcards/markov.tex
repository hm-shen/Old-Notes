% -*- coding:utf-8 -*-
% LATEX PREAMBLE --- needs to be imported manually
\documentclass[11pt]{article}
\special{papersize=3in,5in}
\usepackage[utf8]{inputenc}
\usepackage{amssymb,amsmath}
\pagestyle{empty}
\setlength{\parindent}{0in}
\newcommand{\detail}[1]{{\scriptsize(#1)\par}~}
\newcommand{\refs}[1]{{\scriptsize\textit{Refs: }#1\par}\hfill.}
\newcommand*{\abs}[1]{\left\vert#1\right\vert}

\newcommand{\expect}[1]{\mathbf{E}(#1)}
\newcommand{\var}[1]{\mathbf{Var}(#1)}
\newcommand{\indicator}{\mathbf{1}}

%%% commands that do not need to imported into Anki:
\usepackage{mdframed}
\newcommand*{\tags}[1]{\paragraph{tags: }#1\bigskip}
\newcommand*{\xfield}[1]{\begin{mdframed}\centering #1\end{mdframed}\bigskip}
\newenvironment{field}{}{}
\newcommand*{\xplain}[1]{\begin{mdframed}\texttt{#1}\end{mdframed}\bigskip}
\newenvironment{plain}{\ttfamily}{\par}
\newenvironment{note}{}{}
% END OF THE PREAMBLE
\begin{document}
%
\tags{markov-processes}
\begin{note}
  \xfield{Definition of stopping time}
  \begin{field}
    Let a random process, \(\{X_t, t \in T\}\) defined on
    some probability space and taking values in the set of
    integers, \(D\). Random variable \(\tau\) is said to be a stopping time
    w.r.t. \(\{X_t, t \in T\}\) if the event \(\{\tau = m\}, \forall m
    \in D\) can be determined by \(X_0, X_1, \ldots, X_m\).
  \end{field}
\end{note}
%
\begin{note}
  \xfield{Definition of strong Markov property}
  \begin{field}
    Let \(\{X_t, t \in T\}\) be a Markov process and let \(\tau\) be a
    stopping time w.r.t. \(\{X_t, t \in T\}\). The process \(\{X_t, t
    \in T\}\) satisfies the strong Markov property if \(\forall k =
    1,2,\ldots\),
    \begin{align*}
      P(X_{\tau+k} \in A | X_\tau = x, X_{\tau - 1} = x, \cdots, X_0 =
      x) = P(X_{\tau + k} \in A | X_\tau = x)
    \end{align*}
    (note that when \(\tau\) is a constant, it goes back to Markov property)
  \end{field}
\end{note}
%
\begin{note}
  \xfield{Definition of Markov Chain}
  \begin{field}
    A Markov chain is a Markov process \(\{X_t, t \in T\}\) with
    finite or countably infinite state space. To characterize a Markov
    chain statistically, we need the following
    \begin{align*}
      \{\pi_s(x) & : \forall x \in S, \pi_0(x) = p(X_0 = 0)\} \\
      \{P_t(x, y) & := p(X_{t+1} = y | X_t = x), x, y \in S, t \in T\}
    \end{align*}
    where \(S\) is the state of space of the Markov chain.
  \end{field}
\end{note}
%
\begin{note}
  \xfield{Definition of Random Walk}
  \begin{field}
    Consider \(\{\xi_1, \ldots, \xi_k, \ldots\}\), a collection of
    i.i.d. random variables taking values in a set of integers. Let
    \(X_t\) be a random variable that takes values in the set of
    integers and is independent of \(\{\xi_1, \ldots, \xi_k,
    \ldots\}\), \(\{X_t := X_0 + \xi_1 + \cdots + \xi_t\}\) is called
    a random walk.
  \end{field}
\end{note}
%
\begin{note}
  \xfield{Definition of hitting times}
  \begin{field}
    Considering a Markov Chain, \(\{X_t, t \geq 0\}\) with state space
    \(S\) and transition probability, \(P(x,y) = P(X_{t+1} = y | X_t =
    x), \forall x,y \in S, \forall t\). Let \(A \subset S\), a
    stopping time \(T_A\) w.r.t. \(\{X_t, t = 0,1,2,\ldots\}\) is
    defined as
    \begin{align*}
      T_A := \min \{t > 0: X_t \in A\}.
    \end{align*}
  \end{field}
\end{note}
%
\begin{note}
  \xfield{Properties of Markov Chain}
  \begin{field}
    \begin{enumerate}
    \item \(P^n(x,y) = \sum_{m=1}^n P(T_y = m | X_n = x) P^{n-m}(y,y)\);
    \item If \(a\) is an absorbing state, then
      \begin{align*}
        P(X_n = a | X_0 = x) = P(T_a \leq n | X_0 = x)
      \end{align*}
    \end{enumerate}
  \end{field}
\end{note}
%
\begin{note}
  \xfield{Definition of transient and recurrent states}
  \begin{field}
    Let \(\rho_{xy} := P(T_y < +\infty | X_0 = x), \rho_{yy} = P(T_y <
    +\infty | X_0 = y), x,y \in S\). A state \(y \in S\) is said to be
    recurrent if \(\rho_{yy} = 1\). A state \(y \in S\) is transient
    if \(\rho_{yy} < 1\).
  \end{field}
\end{note}
%
\begin{note}
  \xfield{State and prove the properties of a transient state}
  \begin{field}
    Let \(y \in S\) be a transient state, then
    \begin{align*}
      P(N(y) < +\infty | X_0 = x) & = 1\\
      G(x,y) = \expect{N(y) | X_0 = x} & = \frac{\rho_{xy}}{1 -
      \rho_{yy}}, \forall x \in S
    \end{align*}
    where \(N(y) := \sum_{n=1}^{+\infty} \indicator_{\{y\}}(X_n)\)
  \end{field}
\end{note}
%
\begin{note}
  \xfield{State and prove properties of a recurrent state}
  \begin{field}
    Let \(y \in S\) be a recurrent state, then
    \begin{align*}
      P(N(y) = +\infty | X_0 = y) & = 1 \\
      G(y,y) = \expect{N(y) | X_0 = y} & = + \infty \\
      P(N(y) < +\infty | X_0 = x) & = \rho_{xy}, \forall x \in S
    \end{align*}
    where \(N(y) := \sum_{n=1}^{+\infty} \indicator_{\{y\}}(X_n)\).\\

    If \(\rho_{xy} = 0\), then \(G(x,y) = 0\); if \(\rho_{xy} > 0\),
    then \(G(x,y) = +\infty\).
  \end{field}
\end{note}
\end{document}

%%% Local Variables:
%%% mode: latex
%%% TeX-master: t
%%% End:
