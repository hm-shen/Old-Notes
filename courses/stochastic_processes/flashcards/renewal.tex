% -*- coding:utf-8 -*-
% LATEX PREAMBLE --- needs to be imported manually
\documentclass[11pt]{article}
\special{papersize=3in,5in}
\usepackage[utf8]{inputenc}
\usepackage{amssymb,amsmath}
\pagestyle{empty}
\setlength{\parindent}{0in}
\newcommand{\detail}[1]{{\scriptsize(#1)\par}~}
\newcommand{\refs}[1]{{\scriptsize\textit{Refs: }#1\par}\hfill.}
\newcommand*{\abs}[1]{\left\vert#1\right\vert}

\newcommand{\expect}[1]{\mathbf{E}(#1)}
\newcommand{\var}[1]{\mathbf{Var}(#1)}

%%% commands that do not need to imported into Anki:
\usepackage{mdframed}
\newcommand*{\tags}[1]{\paragraph{tags: }#1\bigskip}
\newcommand*{\xfield}[1]{\begin{mdframed}\centering #1\end{mdframed}\bigskip}
\newenvironment{field}{}{}
\newcommand*{\xplain}[1]{\begin{mdframed}\texttt{#1}\end{mdframed}\bigskip}
\newenvironment{plain}{\ttfamily}{\par}
\newenvironment{note}{}{}
% END OF THE PREAMBLE
\begin{document}
%
\tags{renewal-processes}
\begin{note}
  \xfield{Definition of renewal processes}
  \begin{field}
    A renewal process is a counting process where inter-arrival times,
    \(\tau_j, j = 1,2,\ldots\) between consecutive events are
    i.i.d. with cumulative distribution function \(F_\tau(\cdot)\),
    \(\mu := \expect{\tau}\), \(\tau > 0\) with probability \(1\).
  \end{field}
\end{note}
%
\begin{note}
  \xfield{State and prove three basic properties of renewal processes}
  \begin{field}
    Three properties are:
    \begin{enumerate}
    \item For any fixed finite \(t\), \(P(N_t = +\infty) = 0\).
    \item \(\lim_{t \to +\infty} N_t = +\infty\) with probability \(1\).
    \item \(\lim_{t \to +\infty} \expect{N_t} = +\infty\).
    \end{enumerate}
  \end{field}
\end{note}
%
\begin{note}
  \xfield{State and prove the strong law of renewal processes}
  \begin{field}
    \(\displaystyle \lim_{t \to +\infty} \frac{N_t}{t} =
    \frac{1}{\mu}\) with probability \(1\).
  \end{field}
\end{note}
%
\begin{note}
  \xfield{Distribution of renewal process \(\{N_t, t \geq 0\}\) with
    inter-arrival times \(\tau_n\)}
  \begin{field}
    \(P(N_t = n) = F_n(t) - F_{n+1}(t)\), where
    \(F_n = F_\tau * F_\tau \cdots * F_\tau\)
  \end{field}
\end{note}
%
\begin{note}
  \xfield{State central limit theorem for renewal processes}
  \begin{field}
    Assume that the inter-arrival times, \(\tau_j\) for a renewal
    process, \(\{N_t, t \geq 0\}\) have finite variance
    \(\sigma^2\). Then
    \(\xi_t = \frac{N_t - \frac{t}{\mu}}{\sigma \sqrt{t / \mu^3}} \to
    \mathcal{N}(0,1)\) in distribution when \(t \to +\infty\).
  \end{field}
\end{note}
%
\begin{note}
  \xfield{State and prove the renewal equation}
  \begin{field}
    \(m(t) = F_\tau(t) + \int_0^t m(t-x) F(dx)\)
  \end{field}
\end{note}
%
\begin{note}
  \xfield{State and prove the elementary renewal theorem (\(\tau\) is cont.)}
  \begin{field}
    \(\displaystyle \lim_{t \to +\infty} \frac{\expect{N_t}}{t} =
    \frac{1}{\expect{\tau}}\)
  \end{field}
\end{note}
%
\begin{note}
  \xfield{State and prove Blackwell's Theorem}
  \begin{field}
    If \(\{N_t, t = 0,1,\ldots\}\) has an inter-arrival distribution
    that is no-lattice, then \(\forall \sigma > 0\),
    \begin{align*}
      \lim_{t \to +\infty} [ m(t+\sigma) - m(t) ]
      & = \frac{\sigma}{\mu} \\
      \lim_{t \to +\infty} P(N_{t + \sigma} - N_t = 1)
      & = \frac{\sigma}{\mu} \\
      \lim_{t \to +\infty} P(N_{t+\sigma} - N_t = 0)
      & = 1 - \frac{\sigma}{\mu} + o(\sigma) \\
      \lim_{t \to +\infty} P(N_{t+\sigma} - N_t = 2)
      & = o(\sigma)
    \end{align*}
    If \(\{N_t, t = 0,1,\ldots\}\) has an inter-arrival distribution
    that is lattice with span \(d\), then:
    \begin{align*}
      \lim_{t \to +\infty} [ m(nd) - m((n-1)d)] = \frac{nd}{\mu}
    \end{align*}
  \end{field}
\end{note}
%
\begin{note}
  \xfield{Definition of renewal reward processes and fundamental assumption}
  \begin{field}
    Let \(\{N_t\}, t \geq 0\) be a renewal process, \(\{R(t), t
    \geq0\}\) be a reward process associated with \(\{N_t\}\). We
    assume \(R_t\) at a give time \(t\) depends only on the
    inter-renewal interval containing \(t\), i.e.,
    \begin{align*}
      R_t = R(Z_t, \tau_t)
    \end{align*}
    where \(Z_t := t - T_{N_t}, \tau_t := T_{N_t + 1} - T_{N_t}\).
  \end{field}
\end{note}
%
\begin{note}
  \xfield{Determine \(\displaystyle \lim_{t \to +\infty} \frac{1}{t} \int_0^t R_s
    ds\)}
  \begin{field}
    \begin{align*}
      \lim_{t \to +\infty} \frac{1}{t} \int_0^t R_s ds = \frac{\expect{R_n}}{\mu}
    \end{align*}
    where \(\mu := \expect{\tau}, R_n = \displaystyle \int_{T_{n-1}}^{T_n} R(Z_t,
    \tau_n) d \tau =  \int_0^\tau R(s, \tau_n) ds\)
  \end{field}
\end{note}
%
\begin{note}
  \xfield{Prove the key renewal reward theorem when \(\{N_t\}\) is non-arithmetic}
  \begin{field}
    Let \(R(\tau, Z)\) be a renewal reward function associated with a
    non-arithmetic renewal process \(\{N_t, t \geq 0\}\); let \(r(z) =
    \int_{\tau = z}^{+\infty} R(\tau, z) d f(\tau)\) be a directly
    Riemann integrable function and let \(m(t) :=
    \expect{N_t}\). Assume \(\tau > 0\) with probability \(1\). We
    have the following:
    \begin{align*}
      \displaystyle
      \lim_{t \to +\infty} \expect{R_t} = \lim_{t \to +\infty}
      \int_0^t r(z) d( m(t-z) ) = \frac{1}{\mu} \int_0^{+\infty} r(z) dz
    \end{align*}
  \end{field}

\end{note}
%
\begin{note}
  \xfield{Determine \(\displaystyle \lim_{t \to +\infty}
    \expect{R_t}\)}
  \begin{field}
    By the key renewal reward theorem when \(\{N_t\}\) is
    non-arithmetic, we have the following:
    \begin{align*}
      \displaystyle
      \lim_{t \to +\infty} \expect{R_t} = \frac{\expect{R_n}}{\mu}
    \end{align*}
    where \(\mu := \expect{\tau}\).
  \end{field}
\end{note}
%
\begin{note}
  \xfield{Definition of delayed renewal process}
  \begin{field}
    A delayed renewal process is a special renewal process where
    \(\tau_2, \ldots, \tau_n, \ldots\) are i.i.d. but \(\tau_1\) is
    independent from and has a different distribution then \(\tau_2,
    \ldots\).
  \end{field}
\end{note}
\end{document}

%%% Local Variables:
%%% mode: latex
%%% TeX-master: t
%%% End:
