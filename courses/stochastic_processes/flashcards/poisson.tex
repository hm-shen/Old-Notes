% -*- coding:utf-8 -*-
% LATEX PREAMBLE --- needs to be imported manually
\documentclass[11pt]{article}
\special{papersize=3in,5in}
\usepackage[utf8]{inputenc}
\usepackage{amssymb,amsmath}
\pagestyle{empty}
\setlength{\parindent}{0in}
\newcommand{\detail}[1]{{\scriptsize(#1)\par}~}
\newcommand{\refs}[1]{{\scriptsize\textit{Refs: }#1\par}\hfill.}
\newcommand*{\abs}[1]{\left\vert#1\right\vert}

\newcommand{\expect}[1]{\mathbf{E}(#1)}
\newcommand{\var}[1]{\mathbf{Var}(#1)}

%%% commands that do not need to imported into Anki:
\usepackage{mdframed}
\newcommand*{\tags}[1]{\paragraph{tags: }#1\bigskip}
\newcommand*{\xfield}[1]{\begin{mdframed}\centering #1\end{mdframed}\bigskip}
\newenvironment{field}{}{}
\newcommand*{\xplain}[1]{\begin{mdframed}\texttt{#1}\end{mdframed}\bigskip}
\newenvironment{plain}{\ttfamily}{\par}
\newenvironment{note}{}{}
% END OF THE PREAMBLE
\begin{document}
%
\tags{poisson-processes}
\begin{note}
  \xfield{Definition of Poisson Process}
  \begin{field}
     A counting process \(\{N_t, t \geq \}\) is called a Poisson
     Process with parameter \(\lambda\) if it satisfies the following
     conditions:
     \begin{enumerate}
     \item \(N_0 = 0\);
     \item For any \(0 \leq s < t < +\infty\), the increment \(N_t -
       N_s\) is a Poisson random variable with parameter
       \(\lambda(t-s)\);
     \item \(\{N_t, t \geq 0\}\) is an independent increment process.
     \end{enumerate}
  \end{field}
\end{note}
%
\begin{note}
  \xfield{Six properties of Poisson Processes}
  \begin{field}
  \begin{enumerate}
  \item \(\forall t > 0, N_t\) is a Poisson random varialbe with
    parameter \(\lambda t\);
  \item \(\expect{N_t} = \lambda t, \var{N_t} = \lambda t,
    \expect{N^2_t} = \var{N_t} + \expect{N_t}^2 = \lambda t + (\lambda
    t)^2\)
  \item Consider \(N_{t + \delta} - N_t\) where \(\delta > 0\) is very
    small.
    \begin{align*}
      P(N_{t + \delta} - N_t = 0)
      & = e^{-\lambda \delta} = 1 - \lambda \delta + o(\delta) \\
      P(N_{t + \delta} - N_t = 1)
      & = \lambda \delta e^{-\lambda \delta} = \lambda \delta +
        o(\delta) \\
      P(N_{t + \delta} - N_t \geq 2)
      & = 1 - P(N_{t + \delta} - N_t = 0) - P(N_{t + \delta} - N_t =
        1) = o(\delta)
    \end{align*}
  \item Let \(T_n := \min \{ t > 0: N_t = 0\}\), then \(\{T_n > t\} =
    \{N_t < n\}\). (Applies to all counting processes)
  \item Let \(S_n := T_n - T_{n-1}\), \(S_1, S_2, \ldots, S_n\) are
    i.i.d. and exponentially distributed with parameter
    \(\lambda\). Also note that they obey memoryless property.
  \item The inter-arrival of a Poisson process are i.i.d exponential
    random variables with parameter \(\lambda\) and have momoryless
    property.
  \end{enumerate}
  \end{field}
\end{note}
%
\begin{note}
  \xfield{State and prove that the countable infinite sum of independent Poisson
    Processes is a Poisson Process}
  \begin{field}
    Sum of independent Poisson Processes:\\

    Let \(\{N_i, i = 1,2,\ldots\}\) be a family of independent Poisson
    Processes with respective parameters \(\lambda_i \geq 0, i =
    1,2,\ldots\), Then
    \begin{enumerate}
    \item If \(\lambda := \sum_i^{+\infty} \lambda_i < +\infty\), then
      \(\{N_t = \sum_1^{+\infty} N^i_t, t \geq 0\}\) is a Poisson
      Process with parameter \(\lambda\).
    \item Any two distinct Poisson Process from this family have no
      points in common.
    \end{enumerate}
    \refs{For proof, please refer to notes on EECS 502}
  \end{field}
\end{note}
%
\begin{note}
  \xfield{State and prove Competition Theorem for Poisson Processes}
  \begin{field}
    Competition Theorem:\\

    Let \(\{N_i, i = 1,2,\ldots\}\) be a family of independent Poisson
    Processes with parameters \(\lambda_i, i = 1,2,\ldots\)
    respectively. Assume \(\lambda := \sum_1^{+\infty} \lambda_i <
    +\infty\), define \(N := \sum_1^{+\infty} N_i\) and let \(\tau\)
    be the first event time of Poisson process \(N\) and \(Z\) be the
    index of the Poisson process responsible for it, that is \(\tau\)
    is the first event time of \(N_Z\), then
    \begin{align*}
      P(Z=i, \tau \geq t) = P(Z = i) P(\tau \geq t) =
      \frac{\lambda_i}{\lambda} e^{-\lambda t}
    \end{align*}
    \refs{For proof, please refer to notes on EECS 502}
  \end{field}
\end{note}
\end{document}
