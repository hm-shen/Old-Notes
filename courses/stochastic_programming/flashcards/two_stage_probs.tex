% -*- coding:utf-8 -*-
% LATEX PREAMBLE --- needs to be imported manually
\documentclass[11pt]{article}
\special{papersize=3in,5in}
\usepackage[utf8]{inputenc}
\usepackage{amssymb,amsmath}
\pagestyle{empty}
\setlength{\parindent}{0in}
\newcommand{\detail}[1]{{\scriptsize(#1)\par}~}
\newcommand{\refs}[1]{{\scriptsize\textit{Refs: }#1\par}\hfill.}
\newcommand*{\abs}[1]{\left\vert#1\right\vert}

\newcommand{\expect}[1]{\mathbf{E}(#1)}
\newcommand{\var}[1]{\mathbf{Var}(#1)}
\newcommand{\reals}{\mathbf{R}}

%%% commands that do not need to imported into Anki:
\usepackage{mdframed}
\newcommand*{\tags}[1]{\paragraph{tags: }#1\bigskip}
\newcommand*{\xfield}[1]{\begin{mdframed}\centering #1\end{mdframed}\bigskip}
\newenvironment{field}{}{}
\newcommand*{\xplain}[1]{\begin{mdframed}\texttt{#1}\end{mdframed}\bigskip}
\newenvironment{plain}{\ttfamily}{\par}
\newenvironment{note}{}{}
% END OF THE PREAMBLE
\begin{document}
%
\tags{linear-two-stage-problems}
\begin{note}
  \xfield{Standard linear two-stage problems and dual of the second-stage}
  \begin{field}
    A standard linear two-stage problem has the following form:
    \begin{align*}
      \min_{x \in \reals^n} ~ & c^\top x + \expect{ Q(x, \xi) } \\
      s.t. ~ & Ax = b \\
      & x \succeq 0
    \end{align*}
    where \(Q(x, \xi)\) is the optimal value of the second-stage
    problem:
    \begin{align*}
      \min_{y \in \reals^m} ~ & q^\top y \\
      s.t. ~ & Wy = h - Tx \\
      & y \succeq 0
    \end{align*}
    \(\xi := (q, h, T, W)\) represents the data of the second-stage
    problem. Expectation is taken w.r.t. the distribution of \(\xi\).

    The dual of the second-stage linear programming is shown below:
    \begin{align*}
      \max_{\pi} ~ & \pi^\top (h - Tx) \\
      s.t. ~ & W^\top \pi \preceq q
    \end{align*}
  \end{field}
\end{note}
%
\begin{note}
  \xfield{State and prove the properties of \(Q(\cdot, \xi)\)}
  \begin{field}
    Properties are listed below. Please refer
    \textit{Lectures on Stochastic Programming} for proofs.
    \begin{enumerate}
    \item For any given \(\xi\), the function \(Q(\cdot, \xi)\) is
      convex. Moreover, if the set \(\{\pi : W^\top \pi \preceq\}\) is
      nonempty and second stage problem is feasible, then the function
      \(Q(\cdot, \xi)\) is polyhedral.
    \item Suppose that for given \(x = x_0\) and \(\xi \in \Xi\), the
      value \(Q(x_0, \xi)\) is finite. Then \(Q(\cdot, \xi)\) is
      subdifferentiable at \(x_0\) and
      \begin{align*}
        \partial Q(x_0, \xi) = -T^\top \mathcal{D}(x_0, \xi)
      \end{align*}
      where \(\mathcal{D}(x,\xi) := \text{arg } \max_{\pi \in \Pi(q)}
      \pi^\top (h - Tx)\) is the set of optimal solutions of the dual
      problem.
    \end{enumerate}
  \end{field}
\end{note}
%
\begin{note}
  \xfield{Definitions and conditions under which the two-stage problem
    has fixed recourse, complete recourse, relatively complete
    recourse, simple recourse}
  \begin{field}
    see textbook.
  \end{field}
\end{note}
\end{document}


%%% Local Variables:
%%% mode: latex
%%% TeX-master: t
%%% End:
