% -*- coding:utf-8 -*-
% LATEX PREAMBLE --- needs to be imported manually
\documentclass[11pt]{article}
\special{papersize=3in,5in}
\usepackage[utf8]{inputenc}
\usepackage{amssymb,amsmath}
\pagestyle{empty}
\setlength{\parindent}{0in}
\newcommand{\detail}[1]{{\scriptsize(#1)\par}~}
\newcommand{\refs}[1]{{\scriptsize\textit{Refs: }#1\par}\hfill.}
\newcommand*{\abs}[1]{\left\vert#1\right\vert}

\newcommand{\expect}[1]{\mathbf{E}(#1)}
\newcommand{\var}[1]{\mathbf{Var}(#1)}
\newcommand{\reals}{\mathbf{R}}

%%% commands that do not need to imported into Anki:
\usepackage{mdframed}
\newcommand*{\tags}[1]{\paragraph{tags: }#1\bigskip}
\newcommand*{\xfield}[1]{\begin{mdframed}\centering #1\end{mdframed}\bigskip}
\newenvironment{field}{}{}
\newcommand*{\xplain}[1]{\begin{mdframed}\texttt{#1}\end{mdframed}\bigskip}
\newenvironment{plain}{\ttfamily}{\par}
\newenvironment{note}{}{}
% END OF THE PREAMBLE
\begin{document}
\begin{note}
  \xfield{Definition of Semi-continuity}
  \begin{field}
    Roughly speaking, the function values for arguments near \(x_0\)
    are either close to \(f(x_0)\) or less than \(f(x_0)\).
    We say that \(f\) is upper semi-continuous at \(x_0\) if
    \(\forall \epsilon > 0\) there exists a neighborhood \(U\) of
    \(x_0\) such that \(f(x) \leq f(x_0) + \epsilon\) for all
    \(x \in U\) when \(f(x_0) > -\infty\) and \(f(x)\) tends to
    \(- \infty\) as \(x\) tends towards \(x_0\) when
    \(f(x_0) = -\infty\).
  \end{field}
\end{note}

\end{document}
