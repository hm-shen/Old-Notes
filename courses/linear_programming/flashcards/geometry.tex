% -*- coding:utf-8 -*-
% LATEX PREAMBLE --- needs to be imported manually
\documentclass[11pt]{article}
\special{papersize=3in,5in}
\usepackage[utf8]{inputenc}
\usepackage{amssymb,amsmath}
\pagestyle{empty}
\setlength{\parindent}{0in}
\newcommand{\detail}[1]{{\scriptsize(#1)\par}~}
\newcommand{\refs}[1]{{\scriptsize\textit{Refs: }#1\par}\hfill.}
\newcommand*{\abs}[1]{\left\vert#1\right\vert}

\newcommand{\expect}[1]{\mathbf{E}(#1)}
\newcommand{\var}[1]{\mathbf{Var}(#1)}
\newcommand{\reals}{\mathbf{R}}

%%% commands that do not need to imported into Anki:
\usepackage{mdframed}
\newcommand*{\tags}[1]{\paragraph{tags: }#1\bigskip}
\newcommand*{\xfield}[1]{\begin{mdframed}\centering #1\end{mdframed}\bigskip}
\newenvironment{field}{}{}
\newcommand*{\xplain}[1]{\begin{mdframed}\texttt{#1}\end{mdframed}\bigskip}
\newenvironment{plain}{\ttfamily}{\par}
\newenvironment{note}{}{}
% END OF THE PREAMBLE
\begin{document}
\tags{lp-geometry}
\begin{note}
  \xfield{Definition of extreme points and vertices of a polyhedron}
  \begin{field}
    Let \(P \subset \reals^n\) be a polyhedron. A vector \(x \in P\)
    is an extreme point of \(P\) if we cannot find two vectors
    \(y,z \in P\) both different from \(x\), and a scalar
    \(\lambda \in [0,1]\), such that
    \(x = \lambda y + (1 - \lambda) z\).
    \\

    A vector \(x \in P\) is a vertex of \(P\) if there exists some
    \(c \in \reals^n\) such that \(c^\top x < c^\top y\) for all
    \(y \in P, y \not = x\).
  \end{field}
\end{note}
%
\begin{note}
  \xfield{Three equivalent statement w.r.t. the set of indices of
    constraints active at \(x^* \in \reals^n\)}
  \begin{field}
    Let \(x^*\) be an element of \(\reals^n\) and let \(I = \{ i ~|~
    a^\top_i x^* = b_i\}\) be the set of indices of constraints that
    are active at \(x^*\). Then the following are equivalent:
    \begin{enumerate}
    \item There exist \(n\) vectors in the set \(\{a_i ~|~ i \in
      I\}\), which are linearly independent.
    \item The span of the vectors \(a_i, i \in I\) is all of
      \(\reals^n\), i.e., every elements of \(\reals^n\) can be
      expressed as a linear combination of the vectors
      \(a_i, i \in I\).
    \item The system of equations \(a^\top_i x = b_i, i \in I\) has a
      unique solution.
    \end{enumerate}
  \end{field}
\end{note}

%
\begin{note}
  \xfield{Defintion of basic (feasible) solutions}
  \begin{field}
    Consider a polyhedron \(P \in \reals^n\) defined by linear
    equality and inequality constraints, and let
    \(\bar{x} \in \reals^n\). \(\bar{x}\) is a basic solution if
    \((1)\) All equality constraints are active (i.e., satisfied);
    \((2)\) Among the constraints that are active at \(\bar{x}\),
    there exist \(n\) that are linearly independent.
    \\

    If \(\bar{x}\) is a basic solution that satisfies all of the
    constraints, we say it is a basic feasible solution.
    \\

    Also note that if the number of constraints used to define \(P \in
    \reals^n\), \(m\) is less than \(n\), then there is no basic
    solutions or basic feasible solutions.
  \end{field}
\end{note}
%
\begin{note}
  \xfield{State and prove the relation between extreme points,
    vertices and basic solutions}
  \begin{field}
    Let \(P\) be a nonempty polyhedron and let \(x \in P\). Then,
    \(x\) being a vertex is equivalent to \(x\) being an extreme point
    and to being a basic feasible solution.

    Corollary: Given a finite number of linear inequality constraints,
    there can only be a finite number of basic or basic feasible
    solutions.
  \end{field}
\end{note}
%
\begin{note}
  \xfield{Given \(\{x \in \reals^n ~|~ Ax = b, x \succeq 0\}\), state
    and prove how would you find basic solutions.}
  \begin{field}
    Consider the constraints \(Ax = b\) and \(x \succeq 0\) and assume
    that \(A \in \reals^{m \times n}\) has linear independent rows. A
    vector \(x \in \reals^n\) is a basic solution iff we have \(Ax =
    b\) and there exists indices \(B(1), \ldots, B(m)\) such that:
    \begin{enumerate}
    \item The columns \(A_{B(1)}, \ldots, A_{B(m)}\) are linearly
      independent;
    \item If \(i \not = B(1), \ldots, B(m)\) then \(x_i = 0\).
    \end{enumerate}
    Note that there are actually \(m + n\) constraints on \(x\) in
    total. Method above gives a way to find a way to identify
    corners. Also it is easy to judge whether the basic solution is
    feasible or not since we only need to check whether \(x_B =
    A^{-1}_B b \succeq 0\) or not.
  \end{field}
\end{note}
%
\begin{note}
  \xfield{Definition of adjacent basic solutions}
  \begin{field}
    Two distinct basic solutions are said to be adjacent if there are
    \(n-1\) linearly independent constraints that are active at both
    of them. For standard form problems, we also say two bases are
    adjacent if they share all but one basic column.
  \end{field}
\end{note}
%
\begin{note}
  \xfield{Why do we always assume \(A\) has full row rank?}
  \begin{field}
    Let \(P = \{x ~|~ Ax = b, x \succeq 0\}\) be a nonempty
    polyhedron, where \(A\) is a matrix of dimensions \(m \times n\),
    with rows \(a^\top_1, \ldots, a^\top_m\). Suppose \(\text{rank}(A) = k <
    m\) and that the rows \(a^\top_{i_1}, \ldots, a^\top_{i_k}\) are
    linearly independent. Consider the polyhedron
    \begin{align*}
      Q = \{ x ~|~ a^\top_{i_1} x = b_{i_1}, \ldots, a^\top_{i_k} x =
      b_{i_k}, x \succeq 0\}
    \end{align*}
    Then \(Q = P\).
  \end{field}
\end{note}
%
\begin{note}
  \xfield{Definition of degeneracy and degeneracy in standard form polyhedra}
  \begin{field}
    A basic solution \(x \in \reals^n\) is said to be degenerate if
    more than \(n\) of the constraints are active at \(x\).
    \\

    Consider the standard form polyhedron \(P = \{x \in \reals^n ~|~
    Ax = b, x \succeq 0\}\) and let \(x\) be a basic solution. Let
    \(m\) be the number of rows of \(A\). The vectors \(x\) is
    degenerate basic solution if more than \(n-m\) of the components
    of are zero.
  \end{field}
\end{note}
%
\begin{note}
  \xfield{Three equivalent statements on the existence of extreme
    point and its corollary}
  \begin{field}
    Suppose that the polyhedron \(P = \{ x \in \reals^n ~|~ a^\top_i x
    \geq b_i, i = 1,2,\ldots,m\}\) is nonempty. Then the following are
    equivalent:
    \begin{enumerate}
    \item The polyhedron \(P\) has at least one extreme point.
    \item The polyhedron \(P\) does not contain a line.
    \item There exist \(n\) vetors out of the family \(a_1, \ldots,
      a_m\) which are linearly independent.
    \end{enumerate}
    Corollary: Every nonempty bounded polyhedron and every nonempty
    polyhedron in standard form has at least one basic feasible
    solution.
  \end{field}
\end{note}
%
\begin{note}
  \xfield{State the optimality of extreme points}
  \begin{field}
    Consider the linear programming problem of minimizing \(c^\top x\)
    over a polyhedron \(P\). Suppose that \(P\) has at least one
    extreme point. Then either the optimal cost is \(-\infty\) or
    there exists an extreme point which is optimal.
  \end{field}
\end{note}
%
% \begin{note}
%   \xfield{Definition of subspaces}
%   \begin{field}
%     We call \(\varnothing \not = S \subseteq \reals^n\) a subspace of
%     \(\reals^n\) if \(\forall x, y \in S, \alpha, \beta \in \reals\),
%     \(\alpha x + \beta y \in S\). (Must contain \(0\)!)
%   \end{field}
% \end{note}
\end{document}
%%% Local Variables:
%%% mode: latex
%%% TeX-master: t
%%% End:
