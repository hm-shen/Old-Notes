% -*- coding:utf-8 -*-
% LATEX PREAMBLE --- needs to be imported manually
\documentclass[11pt]{article}
\special{papersize=3in,5in}
\usepackage[utf8]{inputenc}
\usepackage{amssymb,amsmath}
\pagestyle{empty}
\setlength{\parindent}{0in}
\newcommand{\detail}[1]{{\scriptsize(#1)\par}~}
\newcommand{\refs}[1]{{\scriptsize\textit{Refs: }#1\par}\hfill.}
\newcommand*{\abs}[1]{\left\vert#1\right\vert}

\newcommand{\expect}[1]{\mathbf{E}(#1)}
\newcommand{\var}[1]{\mathbf{Var}(#1)}
\newcommand{\reals}{\mathbf{R}}

%%% commands that do not need to imported into Anki:
\usepackage{mdframed}
\newcommand*{\tags}[1]{\paragraph{tags: }#1\bigskip}
\newcommand*{\xfield}[1]{\begin{mdframed}\centering #1\end{mdframed}\bigskip}
\newenvironment{field}{}{}
\newcommand*{\xplain}[1]{\begin{mdframed}\texttt{#1}\end{mdframed}\bigskip}
\newenvironment{plain}{\ttfamily}{\par}
\newenvironment{note}{}{}
% END OF THE PREAMBLE
\begin{document}
\tags{lp-geometry}
\begin{note}
  \xfield{Definition of extreme points and vertices of a polyhedron}
  \begin{field}
    Let \(P \subset \reals^n\) be a polyhedron. A vector \(x \in P\)
    is an extreme point of \(P\) if we cannot find two vectors
    \(y,z \in P\) both different from \(x\), and a scalar
    \(\lambda \in [0,1]\), such that
    \(x = \lambda y + (1 - \lambda) z\).
    \\

    A vector \(x \in P\) is a vertex of \(P\) if there exists some
    \(c \in \reals^n\) such that \(c^\top x < c^\top y\) for all
    \(y \in P, y \not = x\).
  \end{field}
\end{note}
%
\begin{note}
  \xfield{Defintion of basic (feasible) solutions}
  \begin{field}
    Consider a polyhedron \(P \in \reals^n\) defined by linear
    equality and inequality constraints, and let
    \(\bar{x} \in \reals^n\). \(\bar{x}\) is a basic solution if
    \((1)\) All equality constraints are active (i.e., satisfied);
    \((2)\) Among the constraints that are active at \(\bar{x}\),
    there exist \(n\) that are linearly independent.
    \\

    If \(\bar{x}\) is a basic solution that satisfies all of the
    constraints, we say it is a basic feasible solution.
    \\

    Also note that if the number of constraints used to define \(P \in
    \reals^n\), \(m\) is less than \(n\), then there is no basic
    solutions or basic feasible solutions.
  \end{field}
\end{note}
%
\begin{note}
  \xfield{State and prove the relation between extreme points,
    vertices and basic solutions}
  \begin{field}
    Let \(P\) be a nonempty polyhedron and let \(x \in P\). Then,
    \(x\) being a vertex is equivalent to \(x\) being an extreme point
    and to being a basic feasible solution.

    Corollary: Given a finite number of linear inequality constraints,
    there can only be a finite number of basic or basic feasible
    solutions.
  \end{field}
\end{note}
%
\begin{note}
  \xfield{Definition of subspaces}
  \begin{field}
    We call \(\varnothing \not = S \subseteq \reals^n\) a subspace of
    \(\reals^n\) if \(\forall x, y \in S, \alpha, \beta \in \reals\),
    \(\alpha x + \beta y \in S\). (Must contain \(0\)!)
  \end{field}
\end{note}
\end{document}
%%% Local Variables:
%%% mode: latex
%%% TeX-master: t
%%% End:
