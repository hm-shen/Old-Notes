% -*- coding:utf-8 -*-
% LATEX PREAMBLE --- needs to be imported manually
\documentclass[11pt]{article}
\special{papersize=3in,5in}
\usepackage[utf8]{inputenc}
\usepackage{amssymb,amsmath}
\pagestyle{empty}
\setlength{\parindent}{0in}
\newcommand{\detail}[1]{{\scriptsize(#1)\par}~}
\newcommand{\refs}[1]{{\scriptsize\textit{Refs: }#1\par}\hfill.}
\newcommand*{\abs}[1]{\left\vert#1\right\vert}

\newcommand{\expect}[1]{\mathbf{E}(#1)}
\newcommand{\var}[1]{\mathbf{Var}(#1)}
\newcommand{\reals}{\mathbf{R}}

%%% commands that do not need to imported into Anki:
\usepackage{mdframed}
\newcommand*{\tags}[1]{\paragraph{tags: }#1\bigskip}
\newcommand*{\xfield}[1]{\begin{mdframed}\centering #1\end{mdframed}\bigskip}
\newenvironment{field}{}{}
\newcommand*{\xplain}[1]{\begin{mdframed}\texttt{#1}\end{mdframed}\bigskip}
\newenvironment{plain}{\ttfamily}{\par}
\newenvironment{note}{}{}
% END OF THE PREAMBLE
\begin{document}
\tags{lp-simplex}
\begin{note}
  \xfield{Standard form linear programming problem}
  \begin{field}
    The standard form linear programming problem has the following
    form:
    \begin{align*}
      \min ~ & c^\top x \\
      \text{s.t.} ~ & Ax = b \\
      & x \succeq 0
    \end{align*}
    where \(A \in \reals^{m \times n}\) and \(\text{rank}(A) = m\);
    let \(P\) denotes the corresponding feasible set.
  \end{field}
\end{note}
%
\begin{note}
  \xfield{Definition of feasible direction and how does simplex
    determine its feasible direction}
  \begin{field}
    Let \(x \in P\), a vector \(d \in \reals^n\) is said to be a
    feasible direction at \(x\) if there exists a positive scalar
    \(\theta\) for which \(x + \theta d \in P\).
    \\

    Simplex algorithm finds the optimal solution of a LP by searching
    for the best extreme points in \(P\). Given a starting point,
    i.e. a basic feasible solution, we have \(m\) linearly independent
    basic constraints and \(n - m\) nonbasic constraints associated
    with it. The \(j\)th feasible direction is determined by releasing
    the \(j\)th nonbasic constraints among the \(n-m\) constraints in
    total: the \(j\)th feasible solution is the \(1\)d null space of
    \(m\) basic constraints and \(n-m-1\) nonbasic constraints.
  \end{field}
\end{note}
%
\begin{note}
  \xfield{Derive primal simplex the \(j\)th feasible direction and
    corresponding reduce cost}
  \begin{field}
    Given a current basic feasible solution (BFS), \(\begin{bmatrix} x_B \\
      x_N \end{bmatrix}\), simplex does the following:
    \begin{enumerate}
    \item Current BFS satisfies:
      \begin{align*}
        \tilde{A} x :=
        \begin{bmatrix} A_B & A_N \\ 0 & I_{n-m} \end{bmatrix}
        \begin{bmatrix} x_B \\ x_N \end{bmatrix} =
        \begin{bmatrix} b \\ 0 \end{bmatrix}
      \end{align*}
      Release the \(j\)th constraints, \(j \in \{m+1, \ldots, n\}\),
      solve for the null space of the left \((n-1) \times n\) matrix,
      \(\tilde{A}^j\) using equality \(\tilde{A}^j d = 0\), we have:
      \begin{align*}
        \tilde{A}^j d :=
        \begin{bmatrix} A_B & A_N \\ 0 & I^j_{n-m} \end{bmatrix}
        \begin{bmatrix} d_B \\ d_N \end{bmatrix} =
        \begin{bmatrix} b \\ 0 \end{bmatrix}
      \end{align*}
      where \(I^j_{n-m}\) denotes the identity matrix without the
      \(j\)th row. Evidently, the solution of \(d\) should satisfy:
      \begin{align*}
        A_B d_B + A_{N_j} d_{N_j} & = 0 \\
        d_B & = - A_B^{-1} A_{N_j} d_{N_j} \\
        d^j_N & = 0
      \end{align*}
      where \(A_{N_j}\) denotes the \(j\)th column of \(A_N\);
      \(d_{N_j}\) denotes the \(j\)th entry of \(d_N\); \(d^j_N\)
      denotes \(d_N\) with the \(j\)th entry excluded.
    \item Compute the reduction in objective value along the \(j\)th
      feasible direction with with step \(d_{N_j} = 1\):
      \begin{align*}
        \delta & = c^\top \begin{bmatrix} d_B \\ d_N \end{bmatrix}\\
               & = c^\top_B d_B + c_{N_j} \\
               & = c_{N_j} - c^\top_B A_B^{-1} A_{N_j} d_{N_j}
      \end{align*}
    \end{enumerate}
    If all possible feasible direction leads to an increase in the
    objective value, we can say that current BFS is optimal
  \end{field}
\end{note}
%
\begin{note}
  \xfield{State and prove the relationship between reduced costs and
    optimality of basic feasible solution}
  \begin{field}
    Consider a basic feasible solution \(x\) associated with a basis
    matrix \(B\), and let \(\bar{c}\) be the corresponding vector of
    reduced costs.
    \begin{enumerate}
    \item If \(\bar{c} \succeq 0\), then \(x\) is optimal.
    \item If \(x\) is optimal and nondegernate, then \(\bar{c}
      \succeq 0\).
    \end{enumerate}
  \end{field}
\end{note}
%
\begin{note}
  \xfield{Describe the primal simplex method}
  \begin{field}
    Simplex method searches for an optimal solution by moving from one
    basic feasible solution to another, along the edges of the
    feasible set, always in a cost reducing direction.
  \end{field}
\end{note}
\end{document}
%%% Local Variables:
%%% mode: latex
%%% TeX-master: t
%%% End:
