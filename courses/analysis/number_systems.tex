% -*- coding:utf-8 -*-
% LATEX PREAMBLE --- needs to be imported manually
\documentclass[11pt]{article}
\special{papersize=3in,5in}
\usepackage[utf8]{inputenc}
\usepackage{amssymb,amsmath}
\pagestyle{empty}
\setlength{\parindent}{0in}
\newcommand{\detail}[1]{{\scriptsize(#1)\par}~}
\newcommand{\refs}[1]{{\scriptsize\textit{Refs: }#1\par}\hfill.}
\newcommand*{\abs}[1]{\left\vert#1\right\vert}

\newcommand{\expect}[1]{\mathbf{E}(#1)}
\newcommand{\var}[1]{\mathbf{Var}(#1)}
\newcommand{\reals}{\mathbf{R}}
\newcommand{\rationals}{\mathbf{Q}}

%%% commands that do not need to imported into Anki:
\usepackage{mdframed}
\newcommand*{\tags}[1]{\paragraph{tags: }#1\bigskip}
\newcommand*{\xfield}[1]{\begin{mdframed}\centering #1\end{mdframed}\bigskip}
\newenvironment{field}{}{}
\newcommand*{\xplain}[1]{\begin{mdframed}\texttt{#1}\end{mdframed}\bigskip}
\newenvironment{plain}{\ttfamily}{\par}
\newenvironment{note}{}{}
% END OF THE PREAMBLE
\begin{document}
\tags{analysis-number-systems}
%
\begin{note}
  \xfield{Definition of order and ordered sets}
  \begin{field}
    Let \(S\) be a set. An order on \(S\) is a relation, denoted by
    \(<\), with the following two properties:
    \begin{enumerate}
    \item If \(x \in S, y \in S\) then one and only one of the
      statements: \(x < y, ~ x = y, ~ y < x\) is true.
    \item If \(x,y,z \in S\), if \(x < y\) and \(y < z\), then \(x < z\).
    \end{enumerate}
    An ordered set is a set \(S\) in which an order is defined.
  \end{field}
\end{note}
%
\begin{note}
  \xfield{Definition of upper (lower) bound and supremum (infimum)}
  \begin{field}
    Suppose \(S\) is an ordered set, and \(E \subset S\). If
    \(\exists \beta \in S\), s.t. \(x \leq \beta\) for every
    \(x \in E\), we say that \(E\) is bounded above, and call
    \(\beta\) and upper bound \(E\). (lower bound is similarly
    defined)
    \\

    Suppose \(S\) is an ordered set, \(E \subset S\), and \(E\) is
    bounded above. Suppose there exists an \(\alpha \in S\) with the
    following properties:
    \begin{enumerate}
    \item \(\alpha\) is an upper bound of \(E\).
    \item If \(\gamma < \alpha\) then \(\gamma\) is not an upper
      bound of \(E\).
    \end{enumerate}
    Then, \(\alpha\) is called the least upper bound of \(E\) or the
    supremum of \(E\) and we denote it as \(\alpha := \sup E\).
  \end{field}
\end{note}
%
\begin{note}
  \xfield{The least upper bound property of an ordered set}
  \begin{field}
    An ordered set \(S\) is said to have the least upper bound
    property if the following is true:
    \\

    If \(E \subset S\), \(E\) is not empty and \(E\) is bounded above,
    then \(\sup E\) exists in \(S\).
  \end{field}
\end{note}
%
\begin{note}
  \xfield{Prove the existance of the infimum of any subsets of a
    ordered set with least upper bounded property}
  \begin{field}
    Suppose \(S\) is an ordered set with the least upper bound
    property, \(B \subset S\), \(B\) is not empty and is bounded
    below. Let \(L\) be the set of all lower bounds of \(B\). Then
    \(\alpha = \sup L\) exists in \(S\) and \(\alpha = \inf B\)
  \end{field}
\end{note}
%
\begin{note}
  \xfield{Definition of field}
  \begin{field}
    A field is a set \(F\) with two operations, called addition and
    multiplication, which satisfy the following field axioms:
    \begin{itemize}
    \item Axioms for addition
    \item Axioms for multiplication
    \item The distributive law
    \end{itemize}
  \end{field}
\end{note}
%
\begin{note}
  \xfield{Definition of an ordered field}
  \begin{field}
    An ordered field is a field \(F\) which is also an ordered set,
    such that
    \begin{enumerate}
    \item \(x + y < x + z\) if \(x, y, z \in F\) and \(y < z\),
    \item \(xy > 0\) if \(x \in F, y \in F, x > 0\), and \(y > 0\).
    \end{enumerate}
  \end{field}
\end{note}
%
\begin{note}
  \xfield{Existence of real fields}
  \begin{field}
    There exists an ordered field \(R\) which has the least upper
    bound property. Moreover, \(R\) contains \(\rationals\) as its
    subfield.
  \end{field}
\end{note}
%
\begin{note}
  \xfield{State and prove the archimedean property of \(\reals\)}
  \begin{field}
    If \(x \in R, y \in R\) and \(x > 0\), then there is a positive
    integer \(n\) such that \(nx > y\).
  \end{field}
\end{note}
%
\begin{note}
  \xfield{State and prove the relation between \(\rationals\) and
    \(\reals\)}
  \begin{field}
    Rational numbers are dense in real numbers: If
    \(x \in R, y \in R\) and \(x < y\), then there exists a
    \(p \in \rationals\) such that \(x < p < y\).
  \end{field}
\end{note}
%
\begin{note}
  \xfield{State and prove the Cauchy Inequality}
  \begin{field}
    If \(a_1, \ldots, a_n\) and \(b_1, \ldots, b_n\) are complex
    numbers, then
    \begin{align*}
      \| \sum_{j=1}^n a_j \bar{b}_j \|^2 \leq \sum_{j=1}^n \|a_j\|^2
      \sum_{j=1}^n \|b_j\|^2.
    \end{align*}
  \end{field}
\end{note}
\end{document}

%%% Local Variables:
%%% mode: latex
%%% TeX-master: t
%%% End:
